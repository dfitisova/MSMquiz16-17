\documentclass{beamer}

\usepackage{graphicx}
\usepackage[latin1]{inputenc}
\usepackage[T1]{fontenc}
\usepackage[english]{babel}
\usepackage{listings}
\usepackage{xcolor}
\usepackage{eso-pic}
\usepackage{mathrsfs}
\usepackage{url}
\usepackage{amssymb}
\usepackage{amsmath}
\usepackage{multirow}
\usepackage{hyperref}
\usepackage{booktabs}
\usepackage{bbm}
\usepackage{cooltooltips}
\usepackage{colordef}
\usepackage{beamerdefs}
\usepackage{lvblisting}
\usepackage{amsfonts}

\pgfdeclareimage[height=3.5cm]{logobig}{hucaselogo}
\pgfdeclareimage[height=0.7cm]{logosmall}{Figures/mef}

\renewcommand{\titlescale}{1.0}
\renewcommand{\titlescale}{1.0}
\renewcommand{\leftcol}{0.6}


\title[]{Quiz 11: define an exponential family and show that the parameters are asymptotically normal} %%how to make it narrow??


\authora{Daria Fitisova}
\authorb{}
\authorc{}

\def\linka{http://lvb.wiwi.hu-berlin.de}
\def\linkb{http://case.hu-berlin.de}
\def\linkc{}

\institute{Ladislaus von Bortkiewicz Chair of Statistics \\
C.A.S.E. -- Center for Applied Statistics\\
and Economics\\
Humboldt--Universit�t zu Berlin \\}

\hypersetup{pdfpagemode=FullScreen}

\begin{document}

% 0-1
%%%%%%%%%%%%%%%%%%%%%%%%%%%%%%%%%%%%%%%%
\frame[plain]{

\titlepage
}

%%%%%%%%%%%%%%%%%%%%%%%%%%%%%%%%%%%%%%%%
\section{Exponential family}
%%%%%%%%%%%%%%%%%%%%%%%%%%%%%%%%%%%%%%%%

% 1-1
%%%%%%%%%%%%%%%%%%%%%%%%%%%%%%%%%%%%%%%%
\frame{
\frametitle{What is exponential family?}
%is a set of probability distributions of a certain form%
For a vector parameter ${\theta}={({\theta}_{1},\ldots, {\theta}_{d})}^{T}$, a family of distributions is said to belong to the exponential family, if the probability density function (or PMF) can be written as \begin{displaymath}
f(x\mid\theta)=h(x)exp\Big\{ \sum_{i=1}^{s}{\eta}_{i}(\theta){t}_{i}(x)-a(\theta)\Big\}
\end{displaymath}
or more compact
\begin{displaymath}
f(x\mid\theta)=h(x)exp\small\{ \eta{(\theta)}^{T}t(x)-a(\theta) \small\} .
\end{displaymath}
Here, $h(x)$ is the $base$ $measure$, $\eta$($\theta$) is called the $natural$ $parameter$, $t(x)$ is the $sufficient$ $statistic$ and $a(\theta)$ is the $log-partition$ or $log-normalizer$.
}

% 1-2
%%%%%%%%%%%%%%%%%%%%%%%%%%%%%%%%%%%%%%%%
\frame{
\frametitle{Example: Poisson distribution}
\begin{displaymath}
\color{green}
P(x\mid\theta)=h(x)exp\small\{ \eta{(\theta)}^{T}T(x)-a(\theta) \small\}
\end{displaymath}
\begin{displaymath}
P(x\mid\lambda)=\frac{{\lambda}^{x}{e}^{-\lambda}}{x!}
={(x!)}^{-1}exp(-\lambda)exp[xlog(\lambda)]
\end{displaymath}
\begin{displaymath}
={(x!)}^{-1}exp\small\{ xlog\lambda-\lambda \small\}, 
\end{displaymath}
i.e. belonging to the exponential family with
\begin{itemize}
\item[-] $\theta=log\lambda$
\item[-] $h(x)={(x!)}^{-1}$
\item[-] $t(x)=x$
\item[-] $a(\theta)=\lambda={e}^{\theta}$
\end{itemize}
%which is a 1-parameter exponential family with $\theta=log\lambda$, $h(x)={(x!)}^{-1}$, $t(x)=x$ and $a(\theta)=\lambda={e}^{\theta}$%
}

% 1-3
%%%%%%%%%%%%%%%%%%%%%%%%%%%%%%%%%%%%%%%%
\frame{
\frametitle{Distributions in exponential family}
\begin{itemize}
\item Gaussian: $\mathbb{R}^{p}$\\
\item Bernoulli: binary \{0,1\} \\
\item Binomial: counts of success/failure\\
\item Multinomial: categorical\\
\item Exponential: $\mathbb{R}^{+}$\\
\item Poisson: $\mathbb{N}^{+}$\\
\item Gamma: $\mathbb{R}^{+}$\\
\item  ... \\
\end{itemize}
}


% 1-4
%%%%%%%%%%%%%%%%%%%%%%%%%%%%%%%%%%%%%%%%
\frame{
\frametitle{Motivation}
\begin{itemize}
\item provides a general framework for selecting a possible alternative parametrisation of the distribution
\begin{itemize}
\item GLMs
\item Bayesian estimates in machine learning
\end{itemize}
\item a number of important statistics can be derived at one stroke within the framework of exponential family
\end{itemize}
}

% 1-5
%%%%%%%%%%%%%%%%%%%%%%%%%%%%%%%%%%%%%%%%
%The mean and variance of probability distributions are defined as integrals with respect to the distribution. Thus it may come as a surprise that the mean and variance of distributions in the exponential family can be obtained by calculating derivatives. the mean can be obtained by computing a first deriva- tive of the cumulant function A(?) and the variance can be obtained by computing second derivatives of A(?)%
\frame{
\frametitle{Moments of exponential family}
\begin{eqnarray*}
\frac{\partial{a(\theta)}}{\partial{\theta}} & = & \frac{d}{d\theta} \Big\{ log(\int exp \small\{ {\theta}^{T}t(x) \small\} h(x)dx \Big\} \\
						       & = & \frac{\frac{d}{d\theta}\int exp \small\{ {\theta}^{T} t(x) \small\} h(x)dx}{\int exp \small\{ {\theta}^{T}t(x) \small\} h(x)dx} \\ 
                                                         & = & \frac{t(x)h(x)exp \small\{ {\theta}^{T}t(x) \small\} dx}{\int exp \small\{ {\theta}^{T}t(x) \small\} h(x)dx} \\ 
                                                         & = & \frac{t(x)exp \small\{ {\theta}^{T}t(x) \small\} h(x)dx}{ exp \small\{ -a(\theta) \small\} } \\
                                                        & = & \int t(x)exp \small\{ {\theta}^{T}t(x)-a(\theta) \small\} h(x)dx \\
                                                         & = & \mathbb{E}[t(x)] \\
\end{eqnarray*}
}
% 1-6
%%%%%%%%%%%%%%%%%%%%%%%%%%%%%%%%%%%%%%%%
%
\frame{
\frametitle{Moments of exponential family}
Likewise, it can be shown that:
\begin{eqnarray*}
\frac{\partial^2{a(\theta)}}{\partial{\theta}^2} = Var(t(x)) = \mathbb{E}[t(x)^2]- \mathbb{E}[t(x)]^2\\
\end{eqnarray*}
}


% 1-7
%%%%%%%%%%%%%%%%%%%%%%%%%%%%%%%%%%%%%%%%
\frame{
\frametitle{Example: moments of Poisson distribution}
\begin{itemize}
\item  $\theta$=log($\lambda$): canonical parameter
\item a($\theta$)=exp($\theta$) %log-normalizer%
\item $\mathbb{E}$[Y]=a'($\theta$)
\item Var[Y]=a''($\theta$)
\end{itemize}
wherefrom follows that for Poisson distribution yields $\mathbb{E}$[Y]=$\lambda$, Var[Y]=$\lambda$
}

% 1-7
%%%%%%%%%%%%%%%%%%%%%%%%%%%%%%%%%%%%%%%%
\frame{
\frametitle{Overview of some distributions in exponential family}
\begin{figure}
  \includegraphics[width=\linewidth]{Figures/distribs}
\end{figure}

}

%%%%%%%%%%%%%%%%%%%%%%%%%%%%%%%%%%%%%%%%
\section{Asymptotic normality of parameters}
%%%%%%%%%%%%%%%%%%%%%%%%%%%%%%%%%%%%%%%%
% 2-1
%%%%%%%%%%%%%%%%%%%%%%%%%%%%%%%%%%%%%%%%
\frame{
\frametitle{MLE of the mean parameter in exponential family distributions}
\begin{displaymath}
\ell(\theta\mid D)= log \Big({\prod}^{N}_{n=1}h({x}_{n})\Big) + {\theta}^{T}\Big({\sum}^{N}_{n=1}t({x}_{n})\Big)-Na(\theta).
\end{displaymath}
Taking the gradient with respect to $\theta$ gives:
\begin{displaymath}
{\nabla}_{\theta}\ell={\sum}^{N}_{n=1}t({x}_{n})-N{\nabla}_{\theta}a(\theta),
\end{displaymath}
and setting to zero gives:
\begin{displaymath}
{\nabla}_{\theta}a(\hat{\theta})=\frac{1}{N}{\sum}_{n=1}^{N}t({x}_{n}).
\end{displaymath}
Finally, defining $\mu$:=$\mathbb{E}$[t(x)], we obtain:
\begin{displaymath}
\color{blue}
\hat{\mu}_{ML}=\frac{1}{N}{\sum}^{N}_{n=1}t({x}_{n})
\end{displaymath}

}
% 2-2
%%%%%%%%%%%%%%%%%%%%%%%%%%%%%%%%%%%%%%%%
\frame{
\frametitle{Recap: asymptotic normality (AN)}
Assume a sequence of RVs \small\{${X}_{n}$\small\} with mean ${\mu}_{n}$ and variance ${\sigma}_{n}^{2}$.  \small\{${X}_{n}$\small\} is asymptotically normal if
\begin{displaymath}
\frac{{X}_{n}-{\mu}_{n}}{{\sigma}_{n}} \xrightarrow{D} N(0,1)
\end{displaymath}
Intuition: as we get more and more data, averages of random variables behave like normally distributed random variables
}
% 2-3
%%%%%%%%%%%%%%%%%%%%%%%%%%%%%%%%%%%%%%%%
\frame{
\frametitle{AN of MLE in exponential family: Poisson distribution example}
\begin{displaymath}
\color{blue}
\hat{\mu}_{ML}=\frac{1}{N}{\sum}^{N}_{n=1}t({X}_{n})
\end{displaymath}
By the assumption ${X}_{i}$ are i.i.d., and since $t({X}_{i})$=${X}_{i}$\color{blue}
(consequently, $\hat{\mu}_{ML}$=$\bar{{X}_{n}}$)
\color{black}
we can apply the iid version of the Central Limit Theorem directly to $\hat{\mu_{ML}}$. \\%To do this the only condition we need to verify (in addition to the data being iid) is that Var(Xi) < ?%
Since $\mathbb{E}({X}_{i})$=$Var({X}_{i})$=$\theta$ < $\infty$ for all $\theta$ in the parameter space, it follows that
\begin{displaymath}
\sqrt{n}(\hat{\mu}_{ML}-\theta)/\sqrt{\theta}\xrightarrow{D} N(0,1)
\end{displaymath}
%How good this approximation is depends on n and ?, with the approximation improving as either n or ? becomes larger%
}

% 2-3
%%%%%%%%%%%%%%%%%%%%%%%%%%%%%%%%%%%%%%%%
\frame{
\frametitle{AN of MLE in exponential family: general case}
The proof for the general case is quite complicated. Here is the guideline:
\begin{itemize}
\item Continuous mapping theorem
\item Central limit theorem
\item Delta theorem
\end{itemize} 


For more details one can see: http://www.stat.purdue.edu/~dasgupta/mle.pdf
}
\end{document}

